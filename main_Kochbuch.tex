%%
%Dieser Coder wird  mit XeTex übersetzt, ist aber auch leicht auf pdflatex umzustellen

%%
\AtBeginDocument{\newcommand{\version}{Version 0.2}}
\documentclass[fontsize=12pt,headings=small,ngerman]{scrbook}
\usepackage{calc}
\usepackage[a4paper]{geometry}
\geometry{inner=25mm,outer=50mm,top=40mm,bottom=40mm,bindingoffset=1cm,marginpar=55cm}

%% Seiteneinstellungen für die Kapitelseite auf gleichmäßige Ränder ohne Rand für Marginalien setzen

\newcommand\innen{25mm}
\newcommand\aussen{25mm}
\newcommand\oben{40mm}
\newcommand\unten{40mm}

\usepackage{polyglossia}
\setdefaultlanguage[spelling = new, babelshorthands = true]{german}
\usepackage{xltxtra} % Dieses Paket lödt automatisch die folgenden Pakete: fixltx2e, metalogo, xunicode, fontspec
\usepackage{ellipsis, % Korrigiert den Weißraum um Auslassungspunkte
ragged2e, % Ermöglicht Flattersatz mit Silbentrennung % Befehl \RaggedRight
marginnote, % Für bessere Randnotizen mit \marginnote statt \marginline
}
\usepackage{microtype}
\usepackage[autostyle=true,german=guillemets]{csquotes}
\usepackage[right]{eurosym} % Euro-Geldbeträge setzen (mit Euro-Symbol hinter Zahlenwert}
\usepackage{scrlayer-scrpage}
\usepackage{libertine}
\linespread{1.08} % Durchschuss für Libertine-Font erhöhen
\setlength{\skip\footins}{20pt}
\deffootnote[]{1.5em}{1em}{\thefootnotemark\enskip}
\usepackage{url}
\usepackage[bookmarks=true,%
    bookmarksopen=true,%
    bookmarksnumbered=true,%
    pdftitle={Julians Rezepte},%
    pdfauthor={gesammelt und zusammengestellt von Julian Schepperle},
    pdfsubject={Rezepte zum Nachmachen - einfach und lecker},%
    pdfkeywords={Rezepte, Kochen, Backen, Fleisch, Fisch, vegetarisch, vegan, Süßes},%
]{hyperref}
%Farben
\usepackage[cmyk]{xcolor}
%Tabellen
\usepackage{array,multirow,longtable}

\newcolumntype{P}[1]{>{\hspace{0pt}}p{#1}}

\renewcommand{\arraystretch}{1}
\newcommand\aaa{2mm}  % 1.Spalte
\newcommand\bbb{7mm}  % 2.Spalte
\newcommand\ccc{9mm}  % 3.Spalte
\newcommand\ddd{28mm} % 4.Spalte
\newcommand\eee{58mm} % 5.Spalte

\newcounter{step}
\renewcommand*\thestep{\arabic{step}.\,)}
\newcommand\nextstep{\stepcounter{step}\thestep}

\setcounter{tocdepth}{1}
\usepackage{makeidx}
\makeindex

\author{Julian Schepperle}
\title{Köstliche Rezepte}
\subtitle{gesammelt von}
\date{Januar 2017}
\lowertitleback{Diese Sammlung wurde mit Hilfe von  {\KOMAScript} und {\XeLaTeX} gesetzt. \newline\version}

\hyphenation{To-ma-ten-mark wal-nuss-gro-ße Ma-jo-ran}

% Vakatseiten bis auf die Paginierung leer lassen, also mit Seitenstil plain erzeugen (s.S. scrguide.pdf}
\KOMAoptions{cleardoublepage=plain}

\begin{document}

% % Seiteneinstellungen für Titelei, Inhaltsverzeichnis und die Kapitelseite
% % auf gleichmäßige Ränder ohne Rand für Marginalien setzen
\newgeometry{inner=\innen, outer=\aussen, top=\oben, bottom=\unten}
\maketitle
\tableofcontents
\part{Rezepte}
\label{prt:rezepte}

\chapter{Vegetarisch}
\label{cha:vegetarisch}

% % Seiteneinstellungen auf Ränder mit Marginalienspalten setzen
\restoregeometry

\section{Rote-Bete-Risotto}
\label{sec:rote_bet_risotto}

\begin{flushright}
\textsf{4 Portionen, ca. 45\,Minuten}
\end{flushright}


\noindent \marginnote{Einkaufsliste
    \begin{tabular}{p{11mm}<{\raggedleft}P{23mm}<{\raggedright}}
     & Einkaufsliste\\
    3 & Scharlotten\\
    2 & Äbfel\\
\end{tabular}}

Diese Risotto mit Roten Rosen



\setcounter{step}{0}

\begin{longtable}{>{\footnotesize}p{\aaa}>{\sffamily}P{\bbb}<{\raggedleft}
                >{\sffamily}p{\ccc}<{\raggedright}
                >{\sffamily}P{\ddd}<{\raggedright}P{\eee}<{\raggedright}}
                



\nextstep& 2& & Äpfel & vierteln, schälen und entkernen und\\



\end{longtable}

\noindent Den Meeressparkgel
%chapter vegetarisch

% % Seiteneinstellungen für die Kapitelseite auf gleichmäßige Ränder ohne Rand für Marginalien setzen
\newgeometry{inner=\innen, outer=\aussen, top=\oben, bottom=\unten}
\chapter{Fleischgerichte}
\label{cha:fleischgerichte}

% % Seiteneinstellungen auf Ränder mit Marginalienspalten setzen
\restoregeometry

% Ihre Fleischrezepte aus \inclued{fleisch/ihr-rezept}
% chapter fleischgerichte

% % Seiteneinstellungen für die Kapitelseite auf gleichmäßige Ränder ohne Rand für Marginalien setzen
\newgeometry{inner=\innen, outer=\aussen, top=\oben, bottom=\unten}
\chapter{Fisch}
\label{cha:fisch}

% % Seiteneinstellungen auf Ränder mit Marginalienspalten setzen
\restoregeometry

% Ihre Fischrezepte aus \inclued{fisch/ihr-rezept}
% chapter fischgerichte


% % Seiteneinstellungen für die Kapitelseite auf gleichmäßige Ränder ohne Rand für Marginalien setzen
\newgeometry{inner=\innen, outer=\aussen, top=\oben, bottom=\unten}
\chapter{Süßes}
\label{cha:susses}

% % Seiteneinstellungen auf Ränder mit Marginalienspalten setzen
\restoregeometry

% Ihre Backrezepte aus \inclued{suesses/ihr-rezept}
% chapter susses
% part rezepte
% % Seiteneinstellungen für die Kapitelseite auf gleichmäßige Ränder ohne Rand für Marginalien setzen
\newgeometry{inner=\innen, outer=\aussen, top=\oben, bottom=\unten}
\appendix
\part{Anhang}
\label{prt:anhang}
\label{cha:anhang}
\printindex
\end{document}

